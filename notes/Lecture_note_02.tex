\documentclass[12pt, oneside]{article} 
\usepackage{amsmath, amsthm, amssymb, calrsfs, wasysym, verbatim, bbm, color, graphics, geometry}

\geometry{tmargin=.75in, bmargin=.75in, lmargin=.75in, rmargin = .75in}  

\newcommand{\R}{\mathbb{R}}
\newcommand{\C}{\mathbb{C}}
\newcommand{\Z}{\mathbb{Z}}
\newcommand{\N}{\mathbb{N}}
\newcommand{\Q}{\mathbb{Q}}
\newcommand{\Cdot}{\boldsymbol{\cdot}}
\newcommand{\block}[1]{
  \underbrace{\begin{matrix}1 & \cdots & 1\end{matrix}}_{#1}
}
\newtheorem{thm}{Theorem}
\newtheorem{defn}{Definition}
\newtheorem{conv}{Convention}
\newtheorem{rem}{Remark}
\newtheorem{lem}{Lemma}
\newtheorem{cor}{Corollary}


\title{Lecture Note - 02}
\author{Dihui Lai}

\begin{document}

\maketitle
\tableofcontents

\vspace{.25in}

\section{Discrete Random Variables}

\subsection{Probability Distribution/Mass}

A discrete random variable $X$ can take on $k$ different values: $x_1$, $x_2$, ..., $x_k$, each value occurs with probablity $p_1$, $p_2$, ..., $p_k$. The probablity distribution function or probability mass can be denoted as $X: P(x_i)=p_i$

\subsection{Momentum}

Mean/First Momentum:
$$E(X)=\bar{x}=\sum\limits_{i=1}^k x_i p_i$$
n-th Momentum:
$$E(X^n)=\sum\limits_{i=1}^k x_i^n p_i$$
Variance
$$Var(X)=\sum\limits_{i=1}^k (x_i-\bar{x})^2 p_i$$
Mean of a general function $f(x)$
$$E(f(X))=\sum\limits_{i=1}^k f(x_i) p_i$$

\subsection{Bernouli Distribution}
A random variable that follows Bernouli Distribution can have two possible values $1$ or $0$, with probability $p$ and $1-p$, respectively.

\begin{equation}
P(x)=
\begin{cases}
p, &x=1\\
1-p, &x=0
\end{cases}
\end{equation}

The mean of a Bernouli random variable can be calculated as 
$$\bar{x}=1\cdot p+0\cdot(1-p)=p$$

A sample Bernouli Distribution dataset of N samples is a vector composed of 0/1 e.g. $x=[1, 0, 0, 1, ...1]$ and has $N$ elements. Let's assume there are $N_1$ 1s and $N_0$ 0s in the dataset. Numerically, the average of the dataset can be calculated as 

$$\bar{x}=\frac{1+0+0+1...1}{N}=\frac{N_0}{N}=\frac{N_0}{N_0+N_1}$$

Here,
$\frac{N_0}{N_0+N_1}$ is also the occurence probability of 1 in the dataset i.e. $\frac{N_0}{N_0+N_1}\sim p$

\subsection{Binomial Distribution}
Assume we do 3 experiments and each experiment has an outcome of either 1 or 0. In another word, each experiment's outcome follows Bernouli Distribution. A sample 3 bernouli experiment dataset looks like the following
\[
\begin{bmatrix}
    1 & 0  &0 \\
    0 & 0  &1 \\
    0 & 1  &1 \\
    \vdots \\
    0 &0  &0\\
    1 & 0  &0 \\
    0 &1  &1
\end{bmatrix}
\]
The total number of experiments that have an outcome of 1 can be calclulted as the summation of 3 random variables. 
$X$ can have 4 possible values 0, 1, 2, 3 out of 8 configurations
\begin{equation*}
X=
\begin{cases}
x=0, [0, 0, 0]\\
x=1, [1, 0, 0], [0, 1, 0], [0, 0, 1]\\
x=2, [1, 1, 0], [0, 1, 1], [1, 0, 1]\\
x=3, [1, 1, 1]\\
\end{cases}
\end{equation*}
The probability distribution of each value is therefore
\begin{equation*}
X=
\begin{cases}
x=0, p^3\\
x=1, 3p(1-p)^2\\
x=2, 3p^2(1-p)\\
x=3, (1-p)^3\\
\end{cases}
\end{equation*}

In an n-expemriments case, the dataset looks like the following
\[
\underbrace{
\begin{bmatrix}
    1 & 0  &0 &\cdots &1 & 0  &0  \\
    0 & 0  &1 &\cdots &0 & 0  &0 \\
    0 & 1  &1 &\cdots &0 & 0  &0 \\
    \vdots \\
    0 &0  &0 &\cdots &1& 0  &0\\
    1 & 0  &0 &\cdots &0 & 1 &1\\
\end{bmatrix}
}_{n}
\]
The value of $X$ could vary between 0 and n. The probability that $X=k$, i.e. k out n experiments have out come 1, can have $\frac{n!}{k!(n-k)!}$ configurations. The probability distribution is therefore
$$P(k)=\frac{n!}{k!(n-k)!}p^k(1-p)^{n-k}$$
The expectation of $X$ can be calculated as
\begin{align*}
E(X)&=\sum\limits_{k=0}^{n}kp(k)\\
&=\sum\limits_{k=0}^{n}k\frac{n!}{k!(n-k)!}p^k(1-p)^{n-k}\\ 
&=\sum\limits_{k=1}^{n}k\frac{n!}{k!(n-k)!}p^{k}(1-p)^{n-k}, \text{ the k=0 term is eliminated, the summation starts at k=1}\\
&=\sum\limits_{k=1}^{n}\frac{n!}{(k-1)!(n-k)!}p^{k}(1-p)^{n-k}\\
&=np\sum\limits_{k=1}^{n}\frac{(n-1)!}{(k-1)!(n-k)!}p^{k-1}(1-p)^{n-k}\\
&=np\sum\limits_{k=0}^{n-1}\frac{(n-1)!}{(k-1)!(n-k-1)!}p^{k}(1-p)^{n-k-1}, \text{ substitute }k\leftarrow k+1 \text{ and reindex k from 0}\\
&=np
\end{align*}

\section{Poisson Distribution}
\subsection{Probability Distribution}
Consider a random event, the probability of 1 occurence within a unit time is $p$. What's the probability distribution of events occurence within time interval of $\tau$ (e.g. No. of car accidents occurs in a day in MO). 
Assume the probablity of 1 event occurring is p in a unit time period. We then have the following probability distribution within a short interval $dt$, where $dt\rightarrow 0$
\begin{equation*}
P=
\begin{cases}
1-pdt, \text{no event}\\
pdt, \text{1 event}\\
(pdt)^n\approx 0, \text{n events}
\end{cases}
\end{equation*}
The event occured within a time period of $dt$ follows a Bernouli distribution
\begin{equation*}
{\Delta}=
\begin{cases}
0, 1-pdt\\
1, pdt\\
\end{cases}
\end{equation*}
For an period of $\tau$, we can construct a random variable $X$ that is a summation of $N$ random variables $X=\Delta_1+\Delta_2+...\Delta_N$. Each variable on the right hand side follows the Bernouli distribution described above. Becase $dt \rightarrow 0$ we need to have $N\rightarrow \infty$ to have $Ndt=\tau$. In another word, $X$ is the summation of infinite number of Bernouli distribution or a binomial distribution when $N\rightarrow \infty$
$$X=\Delta_1+\Delta_2+...\Delta_i+...$$
The distribution of X can be derived by taking the limit of a Binomial distribution.
\begin{align*}
P(k)&=\lim_{N \rightarrow \infty}\frac{N!}{k!(N-k)!}(pdt)^k(1-pdt)^{N-k}\\
&=\lim_{N \rightarrow \infty}\frac{N!}{k!(N-k)!}(\frac{p\tau}{N})^k(1-\frac{p\tau}{N})^{N-k}\\
&=\lim_{N \rightarrow \infty}\frac{N!}{k!(N-k)!}(\frac{\lambda}{N})^k(1-\frac{\lambda}{N})^{N-k} \text{ where } \lambda=p\tau\\
&=\lim_{N \rightarrow \infty}\frac{N(N-1)...(N-k+1)}{k!}(\frac{\lambda}{N})^k(1-\frac{\lambda}{N})^{N-k}\\
&=\frac{\lambda^k}{k!}e^{-\lambda}
\end{align*}

According to Taylor expasion of a exponential function, the summation of $P(k)$ is 
\begin{align*}
\sum\limits_{k=0}^{\infty} P(k)&=e^{-\lambda}\sum\limits_{k=0}^{\infty}\frac{\lambda^k}{k!}=e^{-\lambda}e^{\lambda}=1
\end{align*}

\subsection{Moments}
\textbf{Mean}: there are many ways to calculate the mean of Poisson distribution. Here we use a trick by taking the derivative of the equation above, $e^{-\lambda}\sum\limits_{k=0}^{\infty}\frac{\lambda^k}{k!}=1$, we then have 
$$\frac{d}{d\lambda}e^{-\lambda}\sum\limits_{k=0}^{\infty}\frac{\lambda^k}{k!}=0$$ 
$$\rightarrow-e^{-\lambda}\sum\limits_{k=0}^{\infty}\frac{\lambda^k}{k!}+e^{-\lambda}\sum\limits_{k=0}^{\infty}\frac{k\lambda^{k-1}}{k!}=0$$
The first of the equation above is nothing but 1, we therefore have
$$-1+e^{-\lambda}\sum\limits_{k=0}^{\infty}\frac{k\lambda^{k-1}}{k!}=0$$
Organize the equation a little bit we have
$$\frac{1}{\lambda}e^{-\lambda}\sum\limits_{k=0}^{\infty}\frac{k\lambda^{k}}{k!}=1$$
By definition,  $E(X)=e^{-\lambda}\sum\limits_{k=0}^{\infty}\frac{k\lambda^{k}}{k!}$, we therefore have
$$\frac{1}{\lambda}E(X)=1 \text{ or } E(X)=\lambda$$ 
\textbf{Variance}: To calculate the variance we use the same trick of derivative. Here, we take the derivative of the equation $E(k)=\sum\limits_{k=0}^{\infty}\frac{k\lambda^{k}}{k!}=\lambda$
\begin{align*}
&\frac{d}{d\lambda}E(k)=\frac{d}{d\lambda}\sum\limits_{k=0}^{\infty}\frac{k\lambda^{k}}{k!}e^{-\lambda}=\frac{d}{d\lambda}\lambda\\
\rightarrow&\sum\limits_{k=0}^{\infty}\frac{k^2\lambda^{k-1}}{k!}e^{-\lambda}-\sum\limits_{k=0}^{\infty}\frac{k\lambda^{k}}{k!}e^{-\lambda}=1\\
\rightarrow&\frac{E(k^2)}{\lambda}=1+E(k)\\
\rightarrow&E(k^2)=(1+\lambda)\lambda
\end{align*}
The variance of X by definition is 
\begin{align*}
Var(X)&=\sum\limits_{k=0}^{\infty}(k-\lambda)^2\frac{k\lambda^{k}}{k!}e^{-\lambda}\\
&=E(k^2)-2E(k)\lambda+\lambda^2\\
&=(1+\lambda)\lambda-2\lambda^2+\lambda^2\\
&=\lambda
\end{align*}


\section{Algebra of Multiple Random Variables}
Assume the value of a random variable $X$ is the summation of n random varaibles, $X_1$, $X_2$, ... $X_n$. It can be denoted as
$$X=X_1+X_2+ ... +X_n$$
The expectation of the random variable can be calculated as 
$$E(X)=E(X_1+X_2+X_3 ... +X_n)=E(X_1)+E(X_2)+E(X_3)...+E(X_n)$$
The variance of the random variable is 
$$Var(X)=Var(X_1+X_2+X_3 ... +X_n)=Var(X_1)+Var(X_2)+Var(X_3)...+Var(X_n)+\sum_{i\neq	j}\mathnormal{Cov}(X_i, X_j)$$
When $X_1$, $X_2$, ... $X_n$ are independent, we have $Cov(X_i, X_j)=0$, therefore,
$$Var(X)=Var(X_1)+Var(X_2)+Var(X_3)...+Var(X_n)$$







\end{document}
\documentclass[12pt, oneside]{article} 
\usepackage{amsmath, amsthm, amssymb, calrsfs, wasysym, verbatim, bbm, color, graphics, geometry}
\usepackage{hyperref}

\geometry{tmargin=.75in, bmargin=.75in, lmargin=.75in, rmargin = .75in}  
\setlength{\parindent}{0in}
\setlength{\parskip}{\baselineskip}%
\setlength{\parindent}{1.5pt}%

\newcommand{\R}{\mathbb{R}}
\newcommand{\C}{\mathbb{C}}
\newcommand{\Z}{\mathbb{Z}}
\newcommand{\N}{\mathbb{N}}
\newcommand{\Q}{\mathbb{Q}}
\newcommand{\Cdot}{\boldsymbol{\cdot}}
\newcommand{\block}[1]{
  \underbrace{\begin{matrix}1 & \cdots & 1\end{matrix}}_{#1}
}
\newtheorem{thm}{Theorem}
\newtheorem{defn}{Definition}
\newtheorem{conv}{Convention}
\newtheorem{rem}{Remark}
\newtheorem{lem}{Lemma}
\newtheorem{cor}{Corollary}



\title{Lecture Notes - 03: Linear Regression}
\author{Dihui Lai}

\begin{document}

\maketitle
\tableofcontents

\vspace{.25in}
\section{Linear Regression}

\subsection{Single Variate Linear Regression}
A simple linear model can be formulated by assuming the target variable is dependent only on one predictor i.e.

$$\hat{y}=\beta_0+\beta_1 x$$

In order to have our estimate as close as to the actual value of $y$, we want to find the $\beta$s that minimize the sum squared error function
\begin{equation}
\epsilon=\sum_{i=1}^n(y^i-\hat{y}^i)^2=\sum_{i=1}^n(y^i-\beta_0-\beta_1 x^i)^2
\end{equation}
i.e.
\begin{equation}
\begin{cases}
\frac{\partial \epsilon}{\partial {\beta}_1}=0\\
\frac{\partial \epsilon}{\partial {\beta}_0}=0

\end{cases}
\end{equation}

\begin{equation}
\Rightarrow
\begin{cases}
\sum\limits_{i=1}^{n}(y^{i}-{\beta_0}-{\beta_1} x^{i})x^{i}=0\\
\sum\limits_{i=1}^{n}(y^{i}-{\beta_0}-{\beta_1} x^{i})=0
\end{cases}
\end{equation}

Sorting the equations to get the solution for $\beta_0$ and $\beta_1$
\begin{equation*}
\Rightarrow
\begin{cases}
{\beta_0}\sum\limits_{i=1}^{n}x^{i}=\sum\limits_{i=1}^{n}y^{i}x^{i}-{\beta_1}\sum\limits_{i=1}^{n} x^{i}x^{i}\\
\sum\limits_{i=1}^{n}{\beta_0}=\sum\limits_{i=1}^{n}y^{i}-{\beta_1}\sum\limits_{i=1}^{n} x^{i}
\end{cases}
\end{equation*}

\begin{equation}
\Rightarrow
\begin{cases}
{\beta_1}=\frac{\sum\limits_{i=1}^{n}y^{i}x^{i}-{\beta_0}\sum\limits_{i=1}^{n}x^{i}}{\sum\limits_{i=1}^{n}x^{i}x^{i}}=\frac{\frac{1}{n}\sum\limits_{i=1}^{n}y^{i}x^{i}-{\beta_0}\bar{x}}{\frac{1}{n}\sum\limits_{i=1}^{n}x^{i}x^{i}}\\
{\beta_0}=\frac{1}{n}(\sum\limits_{i=1}^{n}y^{i}-{\beta_1}\sum\limits_{i=1}^{n} x^{i})=(\bar{y}-\beta_1\bar{x})
\end{cases}
\end{equation}

Substitute $\beta_0$ in to the first equation in equation set (4). We have 
\begin{align*}
{\beta_1}&=\frac{\frac{1}{n}\sum\limits_{i=1}^{n}y^{i}x^{i}-(\bar{y}\bar{x}-\beta_1\bar{x}\bar{x})}{\frac{1}{n}\sum\limits_{i=1}^{n}x^{i}x^{i}}\\
\end{align*}

Solving for $\beta_1$ we have
\begin{equation*}
\beta_1{\frac{1}{n}\sum\limits_{i=1}^{n}x^{i}x^{i}}=\frac{1}{n}\sum\limits_{i=1}^{n}y^{i}x^{i}-\bar{y}\bar{x}+\beta_1\bar{x}\bar{x}
\end{equation*}
\begin{equation*}
\Rightarrow
\beta_1{\frac{1}{n}\sum\limits_{i=1}^{n}x^{i}x^{i}}=\frac{1}{n}\sum\limits_{i=1}^{n}y^{i}x^{i}-\bar{y}\bar{x}+\beta_1\bar{x}\bar{x}
\end{equation*}
\begin{equation*}
\Rightarrow
\beta_1({\frac{1}{n}\sum\limits_{i=1}^{n}x^{i}x^{i}}-\bar{x}\bar{x})=\frac{1}{n}\sum\limits_{i=1}^{n}y^{i}x^{i}-\bar{y}\bar{x}
\end{equation*}
Thus we get the solution of $\beta_1$
\begin{equation}
\beta_1=\frac{\frac{1}{n}\sum\limits_{i=1}^{n}y^{i}x^{i}-\bar{y}\bar{x}}{{\frac{1}{n}\sum\limits_{i=1}^{n}x^{i}x^{i}}-\bar{x}\bar{x}}
\end{equation}
Take a close look it is not hard to find that the numerator is the co-variance of $X$ and $Y$. The denominator is the variance of $X$.

Therefore $\beta_1$ can also be written as 
\begin{equation}
\beta_1=\frac{Cov(X,Y)}{Var(X)}=\rho_{XY}\frac{\sigma_X}{\sigma_Y}
\end{equation}

\subsection{Multivariate Linear Regression}

Assume $y$ is a linear superposition of multiple $x$s, the model for y is then formulated as 
$$\hat{y}=\beta_{0}+\beta_{1}x_1+\beta_{2}x_2+\hdots+\beta_{m}x_m$$
or simply
$$
\hat{y}=\sum\limits_{j=1}^{m}\beta_j x_j=\vec{\beta}\cdot\vec{x}
$$
To estimate $\beta$s that best fit the data, we need to minimize the error
\begin{align}
\epsilon&={\sum\limits_{i=1}^{n}(y^{i}-\hat{y}^{i})^2}\\
&={\sum\limits_{i=1}^{n}(y^{i}-\vec{x}^i\cdot\vec{\beta})^2}
\end{align}
Writing in matrix notation, we have
\begin{align*}
\epsilon&={(y-\hat{y})^T(y-\hat{y})}\\
&=(y-X\beta)^T(y-X\beta)
\end{align*}
here $\beta$ is a $m\times1$ matrix defined as 
$$
\beta=\begin{bmatrix}
    \beta_1\\
    \beta_2\\
	\vdots\\
    \beta_m
\end{bmatrix}
$$

To minimize the error $\epsilon$ we want to the ${\beta}$s satisfy the following equation set:
$$
\frac{\partial \epsilon}{\partial \beta_j}=0, j=1, 2, 3, 4 ... m
$$
Using equation (7), we have 
$$
{\sum\limits_{i=1}^{n}\frac{\partial}{\partial \beta_j}(y^{i}-\hat{y}^{i})=0}
$$
\begin{align}
\sum\limits_{i=1}^{n}(y^{i}-\hat{y}^{i})\frac{\partial\hat{y}^{i}}{\partial \beta_j}=0\\
\Rightarrow
\sum\limits_{i=1}^{n}(y^{i}-\hat{y}^{i})x^{i}_{j}=0
\end{align}

Going from equation (9) to equation (10), we use the fact that $\hat{y}=\vec{x}\cdot\vec{\beta}=\sum\limits_{l=1}^{n}x_l^i\beta_l$ and
\begin{equation}
\frac{\partial{\hat{y}^i}}{\partial{\beta_j}}=\frac{\partial}{\partial \beta_j}\sum\limits_{l=1}^{n}x_l^i\beta_l=x_j^i
\end{equation}

Write equation (10) in matrix format we have

$$
\mathnormal{(y-X \beta)}^{T}\mathnormal{X}=0
$$
or after transposing
$$
\mathnormal{X^{T}}\mathnormal{y}-\mathnormal{X^T}\mathnormal{X}\beta=0
$$

Therefore the $\beta$ that minimize $\epsilon$ has to satisfy the following equation

\begin{equation}
\beta=({X^T}{X})^{-1}{X^{T}}{y}
\end{equation}

Loosely speaking, we can interpreting equation (12) as composed two components the covariance related term $X^{T}y$ and a term that is related to the variance of $X$ i.e. ${X^T}{X}$


\section{Likelihood Function}
\subsection{Definition}
If a set of random variables $Y_1$, $Y_2$ ... $Y_n$ has a joint probability distribution density/mass $f(y_1, y_2, ...y_n;\vec{\theta})$, where $\vec{\theta}$ is a set of parameters, the likelihood function is defined as 
\begin{equation}
L(\vec{\theta})=f(y_1, y_2, ...y_n; \vec{\theta})
\end{equation}

Where $y_1$, $y_2$, ... $y_n$ are values drawn from the random distribution $Y_1$, $Y_2$ ... ,$Y_n$ 

\subsection{Example: Likelihood of Bernoulli Distribution}
Assuming an event has two possible outcomes $y=1$ or $y=0$, with probability $p$ of being $1$, i.e. the outcome follows a Bernoulli distribution. As we learned in lecture 2, the probability mass function is 
\begin{align*}
f(y; p)=
\begin{cases}
p, &y=1\\
1-p,  &y=0
\end{cases}
\end{align*}
Or if you express this in a single formula it is 
$$f(y;p)=p^y(1-p)^y$$
you can verify that when $y=1$, $f(y, p)=p$ and when $y=0$, $f(y, p)=1-p$

When you draw a data point from this distribution, the likelihood that the value being 1 is $f(1, p)=p$ and the value being 0 is $f(0, p)=1-p$

\subsection{Example: Likelihood of Gaussian Distribution}
The probability density function (PDF) for a standard Gaussian distribution is 
$$f(x)=\frac{1}{\sqrt{2\pi}}e^{-\frac{x^2}{2}}$$


When you draw a data point from this distribution, the likelihood that the value being 1 is 
$$f(1, p)=\frac{1}{\sqrt{2\pi}}e^{-\frac{1}{2}}$$

The likelihood that the value being $0.5$ is 
$$f(1, p)=\frac{1}{\sqrt{2\pi}}e^{-\frac{0.5^2}{2}}$$


\section{Maximum Likelihood Estimator of the Multivariate Linear Model}
Assume the target variable $y$ in a dataset is draw from a Gaussian distribution and for the $i$th data point, the Gaussian distribution has a mean $\mu_i$; the variance for all the data points are the same i.e. $\sigma^2$. The Gaussian distribution is therefore
$$f(y, \mu_i, \sigma)=\frac{1}{\sqrt{2\pi}}e^{-\frac{(y-\mu_i)^2}{2\sigma^2}}, i=1, 2, 3, ...n $$

Moreover, assume that the parameter $\mu_i$ can be estimated from the predictors of the $i$th data point: $x_j^i$, where $j=1, 2, 3, ...m$  
\begin{equation}
\mu_i=\sum\limits_{j=1}^{m}\beta_j x_j^i=\vec{\beta}\cdot\vec{x}^i
\end{equation}

The likelihood of $i$th data point being $y^i$ is therefore
$$f(y^i, \mu_i, \sigma)=\frac{1}{\sqrt{2\pi}}e^{-\frac{(y^i-\mu_i)^2}{2\sigma^2}}$$
Assuming all data points are independent we would end up having the total likelihood to be

$$
L=\prod_{i=1}^n f(y^i, \mu_i, \sigma)=(\frac{1}{\sqrt{2\pi}})^n e^{-\sum\limits_{i=1}^n\frac{(y^i-\mu_i)^2}{2\sigma^2}}
$$

The corresponding loglikelihood function is 

$$
\ell=log(L)=-\frac{n}{2}log(2\pi)-\sum\limits_{i=1}^n\frac{(y^i-\mu_i)^2}{2\sigma^2}
$$

The $\beta$s that maximize the loglikelihood function need to satisfy the following condtions
$$
\frac{\partial \ell}{\partial \beta_j}=0, j=1, 2, ..., m
$$
Because $-\frac{n}{2}log(2\pi)$ is constant, subsitute $\mu_i$ with equation (14) we end up having an equation set 

$$\frac{\partial}{\partial \beta_j}\sum\limits_{i=1}^n\frac{(y^i-\vec{\beta}\cdot\vec{x}^i)^2}{2\sigma^2}=0, j=1, 2, ..., m$$

Because $\sigma^2$ is constant and independent of $\beta$, we have 
$$\frac{\partial}{\partial \beta_j}\sum\limits_{i=1}^n {(y^i-\vec{\beta}\cdot\vec{x}^i)^2}=0, j=1, 2, ..., m$$

Notice the summation term is exactly the same as the sum-square-error $\epsilon$. \bf {Hence, in this case, the maximum likelihood estimator (MLE) is equivalent to the OLS estimator!} 
\end{document}
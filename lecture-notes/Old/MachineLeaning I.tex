%!TeX encoding = UTF-8
%!TeX program = xelatex
\documentclass[notheorems, aspectratio=54, tikz,border=10pt,multi]{beamer}
% aspectratio: 1610, 149, 54, 43(default), 32

\usepackage{latexsym}
\usepackage{amsmath,amssymb}
\usepackage{mathtools}
\usepackage{color,xcolor}
\usepackage{graphicx}
\usepackage{algorithm}
\usepackage{algpseudocode}

\usepackage{amsthm}
\DeclareMathOperator*{\argmax}{argmax} % thin space, limits underneath in displays
\usepackage{lmodern} % 解决 font warning
% \usepackage[UTF8]{ctex}
\usepackage{animate} % insert gif

\usepackage{lipsum} % To generate test text 
\usepackage{ulem} % 下划线,波浪线

\usepackage{listings} % display code on slides; don't forget [fragile] option after \begin{frame}

% ----------------------------------------------
% tikx
\usepackage{framed}
\usepackage{fancybox}
\usepackage{tikz}
\usepackage{pgf}
\usetikzlibrary{automata, calc,trees,positioning,arrows,chains,shapes.geometric,%
    decorations.pathreplacing,decorations.pathmorphing,shapes,%
    matrix,shapes.symbols, hobby}
\pgfmathsetseed{1} % To have predictable results
% Define a background layer, in which the parchment shape is drawn
\pgfdeclarelayer{background}
\pgfsetlayers{background,main}

% define styles for the normal border and the torn border
\tikzset{
  normal border/.style={orange!30!black!10, decorate, 
     decoration={random steps, segment length=2.5cm, amplitude=.7mm}},
  torn border/.style={orange!30!black!5, decorate, 
     decoration={random steps, segment length=.5cm, amplitude=1.7mm}}}

% Macro to draw the shape behind the text, when it fits completly in the
% page

\def\parchmentframe#1{
\tikz{
  \node[inner sep=2em] (A) {#1};  % Draw the text of the node
  \begin{pgfonlayer}{background}  % Draw the shape behind
  \fill[normal border] 
        (A.south east) -- (A.south west) -- 
        (A.north west) -- (A.north east) -- cycle;
  \end{pgfonlayer}}}

% Macro to draw the shape, when the text will continue in next page
\def\parchmentframetop#1{
\tikz{
  \node[inner sep=2em] (A) {#1};    % Draw the text of the node
  \begin{pgfonlayer}{background}    
  \fill[normal border]              % Draw the ``complete shape'' behind
        (A.south east) -- (A.south west) -- 
        (A.north west) -- (A.north east) -- cycle;
  \fill[torn border]                % Add the torn lower border
        ($(A.south east)-(0,.2)$) -- ($(A.south west)-(0,.2)$) -- 
        ($(A.south west)+(0,.2)$) -- ($(A.south east)+(0,.2)$) -- cycle;
  \end{pgfonlayer}}}

% Macro to draw the shape, when the text continues from previous page
\def\parchmentframebottom#1{
\tikz{
  \node[inner sep=2em] (A) {#1};   % Draw the text of the node
  \begin{pgfonlayer}{background}   
  \fill[normal border]             % Draw the ``complete shape'' behind
        (A.south east) -- (A.south west) -- 
        (A.north west) -- (A.north east) -- cycle;
  \fill[torn border]               % Add the torn upper border
        ($(A.north east)-(0,.2)$) -- ($(A.north west)-(0,.2)$) -- 
        ($(A.north west)+(0,.2)$) -- ($(A.north east)+(0,.2)$) -- cycle;
  \end{pgfonlayer}}}

% Macro to draw the shape, when both the text continues from previous page
% and it will continue in next page
\def\parchmentframemiddle#1{
\tikz{
  \node[inner sep=2em] (A) {#1};   % Draw the text of the node
  \begin{pgfonlayer}{background}   
  \fill[normal border]             % Draw the ``complete shape'' behind
        (A.south east) -- (A.south west) -- 
        (A.north west) -- (A.north east) -- cycle;
  \fill[torn border]               % Add the torn lower border
        ($(A.south east)-(0,.2)$) -- ($(A.south west)-(0,.2)$) -- 
        ($(A.south west)+(0,.2)$) -- ($(A.south east)+(0,.2)$) -- cycle;
  \fill[torn border]               % Add the torn upper border
        ($(A.north east)-(0,.2)$) -- ($(A.north west)-(0,.2)$) -- 
        ($(A.north west)+(0,.2)$) -- ($(A.north east)+(0,.2)$) -- cycle;
  \end{pgfonlayer}}}

% Define the environment which puts the frame
% In this case, the environment also accepts an argument with an optional
% title (which defaults to ``Example'', which is typeset in a box overlaid
% on the top border
\newenvironment{parchment}[1][Example]{%
  \def\FrameCommand{\parchmentframe}%
  \def\FirstFrameCommand{\parchmentframetop}%
  \def\LastFrameCommand{\parchmentframebottom}%
  \def\MidFrameCommand{\parchmentframemiddle}%
  \vskip\baselineskip
  \MakeFramed {\FrameRestore}
  \noindent\tikz\node[inner sep=1ex, draw=black!20,fill=white, 
          anchor=west, overlay] at (0em, 2em) {\sffamily#1};\par}%
{\endMakeFramed}

% ----------------------------------------------

\mode<presentation>{
    \usetheme{CambridgeUS}
    % Boadilla CambridgeUS
    % default Antibes Berlin Copenhagen
    % Madrid Montpelier Ilmenau Malmoe
    % Berkeley Singapore Warsaw
    \usecolortheme{beaver}
    % beetle, beaver, orchid, whale, dolphin
    \useoutertheme{infolines}
    % infolines miniframes shadow sidebar smoothbars smoothtree split tree
    \useinnertheme{circles}
    % circles, rectanges, rounded, inmargin
}
% 设置 block 颜色
\setbeamercolor{block title}{bg=red!30,fg=white}

\newcommand{\reditem}[1]{\setbeamercolor{item}{fg=red}\item #1}

% 缩放公式大小
\newcommand*{\Scale}[2][4]{\scalebox{#1}{\ensuremath{#2}}}

% 解决 font warning
\renewcommand\textbullet{\ensuremath{\bullet}}

% ---------------------------------------------------------------------
% flow chart
\tikzset{
    >=stealth',
    punktchain/.style={
        rectangle, 
        rounded corners, 
        % fill=black!10,
        draw=white, very thick,
        text width=6em,
        minimum height=2em, 
        text centered, 
        on chain
    },
    largepunktchain/.style={
        rectangle,
        rounded corners,
        draw=white, very thick,
        text width=10em,
        minimum height=2em,
        on chain
    },
    line/.style={draw, thick, <-},
    element/.style={
        tape,
        top color=white,
        bottom color=blue!50!black!60!,
        minimum width=6em,
        draw=blue!40!black!90, very thick,
        text width=6em, 
        minimum height=2em, 
        text centered, 
        on chain
    },
    every join/.style={->, thick,shorten >=1pt},
    decoration={brace},
    tuborg/.style={decorate},
    tubnode/.style={midway, right=2pt},
    font={\fontsize{10pt}{12}\selectfont},
}
% ---------------------------------------------------------------------

% code setting
\lstset{
    language=C++,
    basicstyle=\ttfamily\footnotesize,
    keywordstyle=\color{red},
    breaklines=true,
    xleftmargin=2em,
    numbers=left,
    numberstyle=\color[RGB]{222,155,81},
    frame=leftline,
    tabsize=4,
    breakatwhitespace=false,
    showspaces=false,               
    showstringspaces=false,
    showtabs=false,
    morekeywords={Str, Num, List},
}
% ---------------------------------------------------------------------

%% preamble
\title{Introduction to Machine Leaning I}
% \subtitle{The subtitle}
\author{Dihui Lai}
\institute[WUSTL]{dlai@wustl.edu}


% -------------------------------------------------------------

\begin{document}

%% title frame
\begin{frame}
    \titlepage
\end{frame}


\begin{frame}
CONTENT
\begin{itemize}
\item Supervised Learning v.s. Unsupervised Learning
\item k-Nearest Neighbors
\item Naive Bayes Classifier
\item Overfitting \& Corss-validation
\end{itemize} 
\end{frame}


%% normal frame
\section{Supervised Learning v.s. Unsupervised Learning}

\begin{frame}
\frametitle{Supervised Learning v.s. Unsupervised Learning}
\begin{itemize}
\item \textbf{Supervised Leaning}: a model/algorithm that is built on a dataset that contains both input data and desired outcome. For example, logistic regression; linear regression.
\item \textbf{Unsupervised Leaning}: a model/algorithm that is built on a dataset that contains both input data but no desired outcome. For example, K-mean clustering
\end{itemize}

\end{frame}

\section{k-Nearest Neighbors}
\begin{frame}

\frametitle{Distance Metrics and Geometrics of Data}

The distance between two data points $\vec{x}^i$ and $\vec{x}^j$ could be measured using different metrics:
\begin{itemize}
\item Euclidean distance 
$$d_{ij}=\sqrt{\sum_{k=1}^{n}\left(x_k^{i}-x_k^{j}\right)^2}$$
\item Manhattan distance:
$$d_{ij}=\sum_{k=1}^{n}\mid(x_k^{i}-x_k^{j})\mid$$
\item Winkowski distance:
$$d_{ij}=\left(\sum_{k=1}^{n}(x_k^{i}-x_k^{j})^q\right)^{1/q}$$
\end{itemize}


\end{frame}


\begin{frame}
\frametitle{k-Nearest Neighbors}
\begin{itemize}
\item Non-parametric classification/regression machine learning algorithm. 
\item Very easy to implement but can be useful.
\item A decision is made based on the k closest-points in the training dataset. If k is too small (k = 1), the decision will be very noisy. If k is too large, neighbor includes too many data from other classes.  

\end{itemize}


\end{frame}
\begin{frame}
\frametitle{k-Nearest Neighbors Algorithm}
\begin{itemize}
\item[] Load the training and test data 
\item[] Set the value of k to a reasonable value 
\item[] For each point in test data:
\begin{itemize}
       \item[-] Calculate its distance (pick an appropriate metric like Euclidean, Manhattan etc.) to all training data points
       \item[-] Sort the distances from low to high
       \item[-] Choose the first k points in the training data
       \item[-] Assign a class based on the majority of classes present in the chosen points (or take the weighted average for regression)
       
\end{itemize}

\end{itemize}
\end{frame}

\section{Naive Bayes Classifier}


\begin{frame}

\frametitle{Joint Probability Distribution and Chain Rule}

\begin{itemize}
\item The joint probability distrution of two events can be discribed as $$P(A, B)=P(A\mid B)P(B)=P(B \mid A)P(A)$$

If $A$ and $B$ are two indpendent events, then we have $P(B)=P(B\mid A)$ and $P(A)=P(A\mid B)$

\vspace{0.5cm}

\item \textbf{Chain Rule}: Considering n random events $X_1$, $X_2$, $X_3$ ... $X_n$, their joint probability distribution can be described as
\begin{align*}
&P(X_n, \ldots, X_1)\\
&=P(X_{n}|X_{n-1},\ldots ,X_{1})P(X_{n-1},\ldots ,X_{1})\\
&=P(X_{n}|X_{n-1},\ldots ,X_{1})P(X_{n-1}|X_{n-2},\ldots ,X_{1})P(X_{n-2},\ldots ,X_{1})\\
&=...
\end{align*}
\end{itemize}
\end{frame}


\begin{frame}
\frametitle{Bayes' Theorem}
\textbf{Bayes' Theorem}: given two random variables $X$ and $Y$, the conditional probability of $X$ given $Y$ is expressed as:
$$P(Y\mid X)=\frac {P(X\mid Y)P(Y)}{P(X)}$$

Useful terminologies to interpret the equations: $P(Y)$ is called prior, which is the belif in Y without any other knowledge. $P(Y\mid X)$ is the posterior taking into consideration of X. $P(X\mid Y)$ is the likelihood.\\
\vspace{0.5cm}
In a discrete case, the probability distribution of $X$ can be calculated as
$P(X)=\sum _{i}P(X\mid Y_{i})P(Y_{i})$

\end{frame}


\begin{frame}
\frametitle{Baye's Theorem Example}
\begin{parchment}[Rain in California]
You are planning a trip to california tomorrow. Unfortunately, the weatherman has predicted rain for tomorrow. You know in southern california, it only rains 5 days each year and there is a chance the weather man makes false predictions. You searched on line and find that when it rains, the weatherman correctly forecasts rain of 90\% of the time. When it doesn't rain, he incorrectly forecasts rain 10\% of the time. What is the probability that it will rain tomorrow. 
\end{parchment}
\end{frame}

\begin{frame}
\frametitle{Baye's Theorem Example}
\textbf{Solution}: Denote the event that the weatherman forcast a raining day as F. The probability of rain given weatherman's forcast is $$P(1 \mid F)=\frac{P(1)P(F\mid 1)}{P(F)}$$. 

Given that we have $P(F \mid 1)=0.9$, $P(F\mid 0)=0.1$, $P(1)=\frac{5}{365}$ and $P(0)=1-P(1)= \frac{360}{365}$. $P(F)=P(F|1)P(1)+P(F|0)P(0)$,
Therefore, 
$$P(1 \mid F)=\frac{P(1)P(F\mid 1)}{P(F)}=\frac{\frac{5}{365}\cdot0.9}{0.1109}=0.111$$


\end{frame}

\begin{frame}
\frametitle{Naive Bayes Classifier}
The joint probability distribution of predictor $\vec{x}$ and target variable $y$ can be written as
\begin{align*}
&p(x_1, x_2, ...x_n, y)\\
&=p(x_1|x_2, x_3, ..., y)p(x_2, x_3, ..., y)\\
&=p(x_1|x_2, x_3, ..., y)p(x_2|x_3, x_4, ..., y)p(x_3, x_4, ..., y)\\
&=p(x_1|x_2, x_3, ..., y)p(x_2|x_3, x_4, ..., y)...p(x_{n-1}|x_n, y)p(x_n|y)p(y)
\end{align*}


Assuming features are indepedent of each other but only dpendent on the target variable, then we have
$$
p(x_1, x_2, ...x_n, y)=p(y)\prod\limits_{i=1}^{N}p(x_i|y)
$$
\end{frame}


\begin{frame}
    \frametitle{Naive Bayes Classifier}
Using Bayes' theorem, we can get the conditional probability distribution of the target variable as
$$
p(Y|X)=\frac{p(X,Y)}{p(X)}
$$
Therefore, we have 
$$
p(Y|X)=\frac{p(y)\prod\limits_{i=1}^{N}p(x_i|y)}{\sum\limits_{y}p(y)\prod\limits_{i=1}^{N}p(x_i|y)}
$$

The denominator is constant if the features are know. $p(y)$ and $p(x_i\mid y)$ can be calculated from the data. We need to find the y that maximize the $p(Y\mid X)$, i.e.

$$
\hat{y}=\argmax_{y}p(y)\prod\limits_{i=1}^{N}p(x_i|y)
$$

\end{frame}

\section{}



\end{document}

\documentclass[12pt, oneside]{article} 
\usepackage{amsmath, amsthm, amssymb, calrsfs, wasysym, verbatim, bbm, color, graphics, geometry}
\usepackage{mathtools}

\DeclareMathOperator*{\argmax}{arg\,max}
\DeclareMathOperator*{\argmin}{arg\,min}

\geometry{tmargin=.75in, bmargin=.75in, lmargin=.75in, rmargin = .75in}  
\setlength{\parindent}{0in}
\setlength{\parskip}{\baselineskip}%
\setlength{\parindent}{1.5pt}%

\newcommand{\R}{\mathbb{R}}
\newcommand{\C}{\mathbb{C}}
\newcommand{\Z}{\mathbb{Z}}
\newcommand{\N}{\mathbb{N}}
\newcommand{\Q}{\mathbb{Q}}
\newcommand{\Cdot}{\boldsymbol{\cdot}}
\newcommand{\block}[1]{
  \underbrace{\begin{matrix}1 & \cdots & 1\end{matrix}}_{#1}
}
\newtheorem{thm}{Theorem}
\newtheorem{defn}{Definition}
\newtheorem{conv}{Convention}
\newtheorem{rem}{Remark}
\newtheorem{lem}{Lemma}
\newtheorem{cor}{Corollary}
\usepackage{tikz}  %TikZ central library is called.
\usetikzlibrary{automata,positioning} 
\usepackage{standalone}
\usepackage{pdfpages}

\tikzset{%
  every neuron/.style={
    circle,
    draw,
    minimum size=0.8cm
  },
  neuron missing/.style={
    draw=none, 
    scale=2,
    text height=0.2cm,
    execute at begin node=\color{black}$\vdots$
  },
  arro/.style={
    ->,
    >=latex
  },
  bloque/.style={
    draw,
    minimum height=1cm,
    minimum width=0.5cm
  }  
}


\title{Lecture Note - 09: POS Tagging, HMM, NER}
\author{Dihui Lai}

\begin{document}

\maketitle
\tableofcontents

\vspace{.25in}

\section{Part-of-Speech (POS) Tagging}


An important tagset for English is the 45-tag Penn Treebank tagset. 

\begin{itemize}
\item Label the words in a document using POS tags, e.g. 

The[DT] Itek[NNP] Air[NNP] Boeing[NNP] 737[CD] took[VBD] off[RP] bound[VBN] for[IN] Mashhad[NNP] in[IN] north-eastern[JJ] Iran[NNP].

\item If a word $w$ that could be tagged as $\mathnormal{t_1}$, $\mathnormal{t_2}$, ... $\mathnormal{t_k}$, the probabilities the word has tagged  $\mathnormal{t_i}$ is calculated as
$${p(t_i|w) = \frac{c(w,t_i)}{\sum\limits_{i=1}^{k}c(w, t_i)}}$$

\textbf{This approach does not take the order of the word into consideration!}
\end{itemize}

Provided that we have a sequence of words $\mathnormal{W=w_1, w_2,  ...w_i, ...w_n}$ and we want to figure out the their POS tags $\mathnormal{T= t_1, t_2, ... t_i ...t_n}$

Using Bayes' theorem
$$\mathnormal{P(T | W) = P(W|T) P(T) / P(W) = const \times P(W|T) P(T)}$$

Assume that ${t_i}$ is only dependent on ${t_{i-1}}$ and ${w_{i}}$, we have
\begin{align*}
P(T)&= P(t_1) P(t_2 | t_1) P(t_3 | t_1, t_2) P(t_4 | t_1, t_2, t_3)  ...P(t_n | t_{1}, t_{2}, ... t_{n-1})\\
&=P(t_1) P(t_2 | t_1) P(t_3 | t_2) P(t_4 |  t_3)  ...P(t_n | t_{n-1})
\end{align*}
On the other hand, the conditional probability of seeing a word sequence W given a tag sequence T is 
\begin{align*}
P(W|T) = P(w_1 | t_1) P(w_2 | t_2) P(w_3 | t_3)  ...P(w_n | t_{n})
\end{align*}

In summary, we have 
$$\mathnormal{P(T|W) \approx P(t_1) P(t_2|t_1) ... P(t_{n}|t_{n-1}) P(w_1|t_1) P(w_2|t_2) ... P(w_n|t_n)}$$
Each term on the right hand side of the equation can be calculated as 
$$\mathnormal{P(t_{i}|t_{i-1}) = \frac{c(t_{i-1},t_{i})}{c(t_{i-1})}} \text{ (transition probability)}$$ , 
$$\mathnormal{P(w_i|t_i) = \frac{c(w_i,t_i)}{c(t_i)}} \text{ (emission probability)}$$

where 

$\mathnormal{c(t_i)} = \text{count of } \mathnormal{t_i} \text{ in the corpus}$,  

$ \mathnormal{c(w_i,t_i)} = \text{count of }  \mathnormal{(w_i, t_i)} \text{ in the corpus}$,

$ \mathnormal{c(t_{i-1},t_{i})} = \text{count of }  \mathnormal{(t_{i-1}, t_i)} \text{ in the corpus}$



\section{Hidden Markove Model}

In summary, we have 
$$\mathnormal{P(T|W) \approx P(t_1) P(t_2|t_1) ... P(t_{n}|t_{n-1}) P(w_1|t_1) P(w_2|t_2) ... P(w_n|t_n)}$$
Each term on the right hand side of the equation can be calculated as 
$$\mathnormal{P(t_{i}|t_{i-1}) = \frac{c(t_{i-1},t_{i})}{c(t_{i-1})}} \text{ (transition probability)}$$ , 
$$\mathnormal{P(w_i|t_i) = \frac{c(w_i,t_i)}{c(t_i)}} \text{ (emission probability)}$$

where 

$\mathnormal{c(t_i)} = \text{count of } \mathnormal{t_i} \text{ in the corpus}$,  

$ \mathnormal{c(w_i,t_i)} = \text{count of }  \mathnormal{(w_i, t_i)} \text{ in the corpus}$,

$ \mathnormal{c(t_{i-1},t_{i})} = \text{count of }  \mathnormal{(t_{i-1}, t_i)} \text{ in the corpus}$

The best tag sequence is the sequence that maximize the conditional probability $P(T|W)$ i.e.
\begin{align*}
\mathnormal{(t_1, t_2, ...t_n)} & = \mathnormal{\argmax_{t_1, t_2, ...t_n}P(t_1, t_2..., t_n|w_1, w_2, ...w_n)}\\
&= \mathnormal{\argmax_{t_1, t_2, ...t_n} \prod\limits_{i=1}^n P(w_i |t_i ) P(t_i |t_{i-1})}
\end{align*}

To solve the problem, we can use the Viterbi Algorithms for Hidden Markov Model



\section{Named Entity Recognition}
\begin{itemize}
\item POS tag usually label a single word. However, to understand text, we also need to consider the meaning of a set of words.
\item NER labels words in a texts that are names of things e.g. person, organization, money amount, gene/protein names
\item John (person) Lee (person) is the chief of CBSE (organization). 
\item the output of POS-tag could be used as input to accomplish a NER model.
\end{itemize}

IOB: I-Inside, O-Outside, B-Begin

Example: Alex is going to Los Angeles
\begin{align*}
&\text{Alex I-PER}\\
&\text{is O}\\
&\text{going O}\\
&\text{to O}\\
&\text{Los B-LOC}\\
&\text{Angeles I-LOC}
\end{align*}

\begin{itemize}
\item The state of art algorithms:
\item Useful packages, softwares, services
\end{itemize}

\end{document}


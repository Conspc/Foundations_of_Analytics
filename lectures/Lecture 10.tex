%!TeX encoding = UTF-8
%!TeX program = xelatex
\documentclass[notheorems, aspectratio=54]{beamer}
% aspectratio: 1610, 149, 54, 43(default), 32
\usepackage{hyperref}

 
\usepackage{latexsym}
\usepackage{amsmath,amssymb}
\usepackage{mathtools}
\usepackage{color,xcolor}
\usepackage{graphicx}
\usepackage{algorithm}
\usepackage{amsthm}
\DeclareMathOperator*{\argmax}{argmax} % thin space, limits underneath in displays
\usepackage{lmodern} % 解决 font warning
% \usepackage[UTF8]{ctex}
\usepackage{animate} % insert gif

\usepackage{lipsum} % To generate test text 
\usepackage{ulem} % 下划线,波浪线

\usepackage{listings} % display code on slides; don't forget [fragile] option after \begin{frame}

% ----------------------------------------------
% tikx
\usepackage{framed}
\usepackage{fancybox}
\usepackage{tikz}
\usepackage{pgf}
\usetikzlibrary{automata, calc,trees,positioning,arrows,chains,shapes.geometric,%
    decorations.pathreplacing,decorations.pathmorphing,shapes,%
    matrix,shapes.symbols}
\pgfmathsetseed{1} % To have predictable results
% Define a background layer, in which the parchment shape is drawn
\pgfdeclarelayer{background}
\pgfsetlayers{background,main}

% define styles for the normal border and the torn border
\tikzset{
  normal border/.style={orange!30!black!10, decorate, 
     decoration={random steps, segment length=2.5cm, amplitude=.7mm}},
  torn border/.style={orange!30!black!5, decorate, 
     decoration={random steps, segment length=.5cm, amplitude=1.7mm}}}

% Macro to draw the shape behind the text, when it fits completly in the
% page
\def\parchmentframe#1{
\tikz{
  \node[inner sep=2em] (A) {#1};  % Draw the text of the node
  \begin{pgfonlayer}{background}  % Draw the shape behind
  \fill[normal border] 
        (A.south east) -- (A.south west) -- 
        (A.north west) -- (A.north east) -- cycle;
  \end{pgfonlayer}}}

% Macro to draw the shape, when the text will continue in next page
\def\parchmentframetop#1{
\tikz{
  \node[inner sep=2em] (A) {#1};    % Draw the text of the node
  \begin{pgfonlayer}{background}    
  \fill[normal border]              % Draw the ``complete shape'' behind
        (A.south east) -- (A.south west) -- 
        (A.north west) -- (A.north east) -- cycle;
  \fill[torn border]                % Add the torn lower border
        ($(A.south east)-(0,.2)$) -- ($(A.south west)-(0,.2)$) -- 
        ($(A.south west)+(0,.2)$) -- ($(A.south east)+(0,.2)$) -- cycle;
  \end{pgfonlayer}}}

% Macro to draw the shape, when the text continues from previous page
\def\parchmentframebottom#1{
\tikz{
  \node[inner sep=2em] (A) {#1};   % Draw the text of the node
  \begin{pgfonlayer}{background}   
  \fill[normal border]             % Draw the ``complete shape'' behind
        (A.south east) -- (A.south west) -- 
        (A.north west) -- (A.north east) -- cycle;
  \fill[torn border]               % Add the torn upper border
        ($(A.north east)-(0,.2)$) -- ($(A.north west)-(0,.2)$) -- 
        ($(A.north west)+(0,.2)$) -- ($(A.north east)+(0,.2)$) -- cycle;
  \end{pgfonlayer}}}

% Macro to draw the shape, when both the text continues from previous page
% and it will continue in next page
\def\parchmentframemiddle#1{
\tikz{
  \node[inner sep=2em] (A) {#1};   % Draw the text of the node
  \begin{pgfonlayer}{background}   
  \fill[normal border]             % Draw the ``complete shape'' behind
        (A.south east) -- (A.south west) -- 
        (A.north west) -- (A.north east) -- cycle;
  \fill[torn border]               % Add the torn lower border
        ($(A.south east)-(0,.2)$) -- ($(A.south west)-(0,.2)$) -- 
        ($(A.south west)+(0,.2)$) -- ($(A.south east)+(0,.2)$) -- cycle;
  \fill[torn border]               % Add the torn upper border
        ($(A.north east)-(0,.2)$) -- ($(A.north west)-(0,.2)$) -- 
        ($(A.north west)+(0,.2)$) -- ($(A.north east)+(0,.2)$) -- cycle;
  \end{pgfonlayer}}}

% Define the environment which puts the frame
% In this case, the environment also accepts an argument with an optional
% title (which defaults to ``Example'', which is typeset in a box overlaid
% on the top border
\newenvironment{parchment}[1][Example]{%
  \def\FrameCommand{\parchmentframe}%
  \def\FirstFrameCommand{\parchmentframetop}%
  \def\LastFrameCommand{\parchmentframebottom}%
  \def\MidFrameCommand{\parchmentframemiddle}%
  \vskip\baselineskip
  \MakeFramed {\FrameRestore}
  \noindent\tikz\node[inner sep=1ex, draw=black!20,fill=white, 
          anchor=west, overlay] at (0em, 2em) {\sffamily#1};\par}%
{\endMakeFramed}

% ----------------------------------------------

\mode<presentation>{
    \usetheme{CambridgeUS}
    % Boadilla CambridgeUS
    % default Antibes Berlin Copenhagen
    % Madrid Montpelier Ilmenau Malmoe
    % Berkeley Singapore Warsaw
    \usecolortheme{beaver}
    % beetle, beaver, orchid, whale, dolphin
    \useoutertheme{infolines}
    % infolines miniframes shadow sidebar smoothbars smoothtree split tree
    \useinnertheme{circles}
    % circles, rectanges, rounded, inmargin
}
% 设置 block 颜色
\setbeamercolor{block title}{bg=red!30,fg=white}

\newcommand{\reditem}[1]{\setbeamercolor{item}{fg=red}\item #1}

% 缩放公式大小
\newcommand*{\Scale}[2][4]{\scalebox{#1}{\ensuremath{#2}}}

% 解决 font warning
\renewcommand\textbullet{\ensuremath{\bullet}}

% ---------------------------------------------------------------------
% flow chart
\tikzset{
    >=stealth',
    punktchain/.style={
        rectangle, 
        rounded corners, 
        % fill=black!10,
        draw=white, very thick,
        text width=6em,
        minimum height=2em, 
        text centered, 
        on chain
    },
    largepunktchain/.style={
        rectangle,
        rounded corners,
        draw=white, very thick,
        text width=10em,
        minimum height=2em,
        on chain
    },
    line/.style={draw, thick, <-},
    element/.style={
        tape,
        top color=white,
        bottom color=blue!50!black!60!,
        minimum width=6em,
        draw=blue!40!black!90, very thick,
        text width=6em, 
        minimum height=2em, 
        text centered, 
        on chain
    },
    every join/.style={->, thick,shorten >=1pt},
    decoration={brace},
    tuborg/.style={decorate},
    tubnode/.style={midway, right=2pt},
    font={\fontsize{10pt}{12}\selectfont},
}
% ---------------------------------------------------------------------

% code setting
\lstset{
    language=C++,
    basicstyle=\ttfamily\footnotesize,
    keywordstyle=\color{red},
    breaklines=true,
    xleftmargin=2em,
    numbers=left,
    numberstyle=\color[RGB]{222,155,81},
    frame=leftline,
    tabsize=4,
    breakatwhitespace=false,
    showspaces=false,               
    showstringspaces=false,
    showtabs=false,
    morekeywords={Str, Num, List},
}

% ---------------------------------------------------------------------

%% preamble
\title{Lecture 10}
% \subtitle{The subtitle}
\author{Dihui Lai}
\institute[WUSTL]{dlai@wustl.edu}

% -------------------------------------------------------------

\begin{document}

%% title frame
\begin{frame}
    \titlepage
\end{frame}

\section{Document Representation}



\begin{frame}
\frametitle{Document Representations}
A document can be represented by the words and the number of their occurence using term-document matrxix
\begin{center}
 \begin{tabular}{c c c c c} 
 \hline
 Words & As You Like It& Twelfth Night& Ulius Caesar& Henry V\\ [0.5ex] 
 \hline\hline
 battle & 1 & 0 & 7 &13 \\ 
 \hline
 good & 114 & 80 &62 & 89\\
 \hline
 fool & 36 & 58 & 1 & 5 \\
 \hline
 wit & 20 & 15 & 2 & 3 \\
 \hline
\end{tabular}
\end{center}

\end{frame}


\begin{frame}
\frametitle{tf-idf Weighted Measure}
Certain words are more common in all documents e.g. the, it, they. The less frequent like "litigation" might be more important that the more frequent word "good". The tf-idf algorithms can be used to adjust the intuition.
$$
w_{t,d} = tf_{t,d}\times idf_{t}
$$

Here,
$$
tf_{t, d}=1 + log_{10}(count(t, d)) \text{ if } count(t, d) > 0, \text{ else } 0
$$
With $count(t, d)$  the number of occurence of the term $t$ in the document $d$

$$
idf_{t}=log_{10}\left(\frac{N}{df_t}\right)
$$ 
N is the total number of documents in the collection, and $df_t$ is the number of documents in which term t occurs. If word good appears in all collected document, $idf_t=0$

\end{frame}

\begin{frame}
\frametitle{Document/Paragraph to Vector Representation}

\begin{itemize}
\item Average of the word embedding: take the mean of the vector embedding of the words in the document. 
\item LDA (Latent Dirichlet Allocation) [David Blei, Andrew Ng, and Michael I. Jordan, 2003]
\item Distributed Memory version of Paragraph Vector (PV-DM)
\item ...
\end{itemize}

\end{frame}


\begin{frame}
\frametitle{Document Classification, Sentiment Analysis}
\begin{itemize}
\item Document classification: use machine learning to classify a document type, novel, news, entertainment etc. 
\item Sentiment Analysis: given a document/paragraph, infer the underlying sentiment (positive/negative, like/dislike)
\end{itemize}
\end{frame}


\begin{frame}
\frametitle{Support Vector Machine (SVM)}

\begin{itemize}
\item SVM is a classification/regression model that use hyperplane to categorize data points in high dimension space. 
\item Loss function: $\mathnormal{L=\frac{1}{2}|w|^2}$ when $\mathnormal{y^i (\vec{w}\cdot \vec{x}^i+b)\geq 1}$
\item or equivalently  $\mathnormal{L'=\frac{1}{2}|w|^2-\sum\limits_{i=1}^N \alpha_i (y^i (\vec{w}\cdot \vec{x}^i+b)-1)}$ 
\item Optimization: $\mathnormal{\frac{\partial L'}{\partial w_i}=0}$
\end{itemize}

\end{frame}


\begin{frame}
\frametitle{Other NLP Challenges and Transfer Learning}

\begin{itemize}
\item Question and answer, e.g. SQuAD (the Stanford Question Answering Dataset)
\item Machine translation
\item Multi-lingual NLP
\item Transfer learning e.g. BERT (\url {https://arxiv.org/abs/1810.04805})
\end{itemize}

\end{frame}

%% normal frame
\section{Regular Expression}

\begin{frame}
   \frametitle{Regular Expression}
\begin{itemize}
\item Algebraic notation for characterizing a set of strings.
\item Useful for searching in corpus of texts
\item Python example: str = "The rain in Spain"; x = re.findall("Spain", str)
\end{itemize}
\end{frame}

\begin{frame}
   \frametitle{Basic Regular Expression}
\begin{itemize}
\item[-][]: Used to indicate a set of characters e.g. [abc] will match 'a', 'b', or 'c'.
\item[-]'\textbackslash d': matches any decimal digit; equivalent to the set [0-9].
\item[-] '+': match 1 or more repetitions of the preceding RE
\item[-]'\textbackslash D': anything but a number (a non-digit)	
\item[-] '\textbackslash s': space (tab,space,newline etc.)
\item[-] '\textbackslash S': anything but a space e.g. '\textbackslash S+@\textbackslash S+' represents email address
\item[-]'\textasciicircum': This expression matches the start of a string
\end{itemize}

\end{frame}
\begin{frame}[fragile]
\frametitle{Regular Expression in Python}
re.search(regex, text) returns a match object when the pattern is found or not match if the pattern is not found.
\begin{lstlisting}
import re
m = re.search('hello', 'hello world, hello all, good afternoon')
print m.group(0)
#hello
\end{lstlisting}

re.findall(regex, text) will return a list of all the matches.
\begin{lstlisting}
m = re.findall('hello', 'hello world, hello all, good afternoon')
print m
#['hello', 'hello']
\end{lstlisting}
\end{frame}

\end{document}

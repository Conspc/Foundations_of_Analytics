%!TeX encoding = UTF-8
%!TeX program = xelatex
\documentclass[notheorems, aspectratio=54]{beamer}
% aspectratio: 1610, 149, 54, 43(default), 32
\usepackage{pgfplots}

\usepackage{latexsym}
\usepackage{amsmath,amssymb}
\usepackage{mathtools}
\usepackage{color,xcolor}
\usepackage{graphicx}
\usepackage{algorithm}
\usepackage{amsthm}
\usepackage{lmodern} % 解决 font warning
% \usepackage[UTF8]{ctex}
\usepackage{animate} % insert gif

\usepackage{lipsum} % To generate test text 
\usepackage{ulem} % 下划线,波浪线

\usepackage{listings} % display code on slides; don't forget [fragile] option after \begin{frame}

\usepackage{tkz-euclide}

% arrow and line for 'tkzPointShowCoord'
\makeatletter
\tikzset{arrow coord style/.style={%
    densely dashed,
    \tkz@euc@linecolor,
    %>=stealth',
    %->,
    }}
    \tikzset{xcoord style/.style={%
    \tkz@euc@labelcolor,
    font=\normalsize,text height=1ex,
    inner sep = 0pt,
    outer sep = 0pt,
    fill=\tkz@fillcolor,
    below=6pt
    }} 
\tikzset{ycoord style/.style={%
    \tkz@euc@labelcolor,
    font=\normalsize,text height=1ex, 
    inner sep = 0pt,
    outer sep = 0pt, 
    fill=\tkz@fillcolor,
    left=6pt
    }}  
\makeatother
% ----------------------------------------------
% tikx
\usepackage{framed}
\usepackage{tikz}
\usepackage{pgf}
\usetikzlibrary{automata, calc,trees,positioning,arrows,chains,shapes.geometric,%
    decorations.pathreplacing,decorations.pathmorphing,shapes,%
    matrix,shapes.symbols}
\pgfmathsetseed{1} % To have predictable results
% Define a background layer, in which the parchment shape is drawn
\pgfdeclarelayer{background}
\pgfsetlayers{background,main}

% define styles for the normal border and the torn border
\tikzset{
  normal border/.style={orange!30!black!10, decorate, 
     decoration={random steps, segment length=2.5cm, amplitude=.7mm}},
  torn border/.style={orange!30!black!5, decorate, 
     decoration={random steps, segment length=.5cm, amplitude=1.7mm}}}

% Macro to draw the shape behind the text, when it fits completly in the
% page
\def\parchmentframe#1{
\tikz{
  \node[inner sep=2em] (A) {#1};  % Draw the text of the node
  \begin{pgfonlayer}{background}  % Draw the shape behind
  \fill[normal border] 
        (A.south east) -- (A.south west) -- 
        (A.north west) -- (A.north east) -- cycle;
  \end{pgfonlayer}}}

% Macro to draw the shape, when the text will continue in next page
\def\parchmentframetop#1{
\tikz{
  \node[inner sep=2em] (A) {#1};    % Draw the text of the node
  \begin{pgfonlayer}{background}    
  \fill[normal border]              % Draw the ``complete shape'' behind
        (A.south east) -- (A.south west) -- 
        (A.north west) -- (A.north east) -- cycle;
  \fill[torn border]                % Add the torn lower border
        ($(A.south east)-(0,.2)$) -- ($(A.south west)-(0,.2)$) -- 
        ($(A.south west)+(0,.2)$) -- ($(A.south east)+(0,.2)$) -- cycle;
  \end{pgfonlayer}}}

% Macro to draw the shape, when the text continues from previous page
\def\parchmentframebottom#1{
\tikz{
  \node[inner sep=2em] (A) {#1};   % Draw the text of the node
  \begin{pgfonlayer}{background}   
  \fill[normal border]             % Draw the ``complete shape'' behind
        (A.south east) -- (A.south west) -- 
        (A.north west) -- (A.north east) -- cycle;
  \fill[torn border]               % Add the torn upper border
        ($(A.north east)-(0,.2)$) -- ($(A.north west)-(0,.2)$) -- 
        ($(A.north west)+(0,.2)$) -- ($(A.north east)+(0,.2)$) -- cycle;
  \end{pgfonlayer}}}

% Macro to draw the shape, when both the text continues from previous page
% and it will continue in next page
\def\parchmentframemiddle#1{
\tikz{
  \node[inner sep=2em] (A) {#1};   % Draw the text of the node
  \begin{pgfonlayer}{background}   
  \fill[normal border]             % Draw the ``complete shape'' behind
        (A.south east) -- (A.south west) -- 
        (A.north west) -- (A.north east) -- cycle;
  \fill[torn border]               % Add the torn lower border
        ($(A.south east)-(0,.2)$) -- ($(A.south west)-(0,.2)$) -- 
        ($(A.south west)+(0,.2)$) -- ($(A.south east)+(0,.2)$) -- cycle;
  \fill[torn border]               % Add the torn upper border
        ($(A.north east)-(0,.2)$) -- ($(A.north west)-(0,.2)$) -- 
        ($(A.north west)+(0,.2)$) -- ($(A.north east)+(0,.2)$) -- cycle;
  \end{pgfonlayer}}}
% Define the environment which puts the frame
% In this case, the environment also accepts an argument with an optional
% title (which defaults to ``Example'', which is typeset in a box overlaid
% on the top border
\newenvironment{parchment}[1][Example]{%
  \def\FrameCommand{\parchmentframe}%
  \def\FirstFrameCommand{\parchmentframetop}%
  \def\LastFrameCommand{\parchmentframebottom}%
  \def\MidFrameCommand{\parchmentframemiddle}%
  \vskip\baselineskip
  \MakeFramed {\FrameRestore}
  \noindent\tikz\node[inner sep=1ex, draw=black!20,fill=white, 
          anchor=west, overlay] at (0em, 2em) {\sffamily#1};\par}%
{\endMakeFramed}

% ----------------------------------------------

\mode<presentation>{
    \usetheme{CambridgeUS}
    % Boadilla CambridgeUS
    % default Antibes Berlin Copenhagen
    % Madrid Montpelier Ilmenau Malmoe
    % Berkeley Singapore Warsaw
    \usecolortheme{beaver}
    % beetle, beaver, orchid, whale, dolphin
    \useoutertheme{infolines}
    % infolines miniframes shadow sidebar smoothbars smoothtree split tree
    \useinnertheme{circles}
    % circles, rectanges, rounded, inmargin
}
% 设置 block 颜色
\setbeamercolor{block title}{bg=red!30,fg=white}

\newcommand{\reditem}[1]{\setbeamercolor{item}{fg=red}\item #1}

% 缩放公式大小
\newcommand*{\Scale}[2][4]{\scalebox{#1}{\ensuremath{#2}}}

% 解决 font warning
\renewcommand\textbullet{\ensuremath{\bullet}}

% ---------------------------------------------------------------------
% flow chart
\tikzset{
    >=stealth',
    punktchain/.style={
        rectangle, 
        rounded corners, 
        % fill=black!10,
        draw=white, very thick,
        text width=6em,
        minimum height=2em, 
        text centered, 
        on chain
    },
    largepunktchain/.style={
        rectangle,
        rounded corners,
        draw=white, very thick,
        text width=10em,
        minimum height=2em,
        on chain
    },
    line/.style={draw, thick, <-},
    element/.style={
        tape,
        top color=white,
        bottom color=blue!50!black!60!,
        minimum width=6em,
        draw=blue!40!black!90, very thick,
        text width=6em, 
        minimum height=2em, 
        text centered, 
        on chain
    },
    every join/.style={->, thick,shorten >=1pt},
    decoration={brace},
    tuborg/.style={decorate},
    tubnode/.style={midway, right=2pt},
    font={\fontsize{10pt}{12}\selectfont},
}
% ---------------------------------------------------------------------

% code setting
\lstset{
    language=C++,
    basicstyle=\ttfamily\footnotesize,
    keywordstyle=\color{red},
    breaklines=true,
    xleftmargin=2em,
    numbers=left,
    numberstyle=\color[RGB]{222,155,81},
    frame=leftline,
    tabsize=4,
    breakatwhitespace=false,
    showspaces=false,               
    showstringspaces=false,
    showtabs=false,
    morekeywords={Str, Num, List},
}

% ---------------------------------------------------------------------

%% preamble
\title{Foundations of Analytics: Lecture 1}
% \subtitle{The subtitle}
\author{Dihui Lai}
\institute[WUSTL]{dlai@wustl.edu}

% -------------------------------------------------------------

\begin{document}

%% title frame
\begin{frame}
    \titlepage
\end{frame}

%% normal frame
\begin{frame}
CONTENT
\begin{itemize}
\item Elementary Data Analytics
\item Data Analytics in Science, Finance, Insurance, Health Care etc. 
\item Mathematics of Data Analytics
\item Review of Linear Algerbra 
\item Overview of Statistic Models and Machine Learning
\item Computational Tools, Library, Packages, Softwares
\end{itemize}
\end{frame}

\section{Elementary Data Analytics}
\begin{frame}
\frametitle{Elementary Data Problem}
\begin{table}[]
\begin{tabular}{llll}
Person &Pants & Socks &Cost\\
\hline
John	&1	&1 &\$23\\
David	&1	&2  &\$26\\
Lisa	&1	&1  &\$24\\
\hline
\end{tabular}
\end{table}
What are the price of pants and socks?
\end{frame}



\frametitle{Elementary Data Problem}
\begin{frame}
\frametitle{Elementary Data Problem}
\begin{center}
Solution
\end{center}

\end{frame}


\begin{frame}
\frametitle{Mathematics of Data Analytics}
\begin{itemize}
\item Linear Algerbra (Handle Multi-dimension Space)
\item Statistics (Useful Description)
\item Calculus (Optimal Solution/Model/Function)
\item Program(Key Numeric Solution)
\end{itemize}
\end{frame}


\section{Data Analytics in Science, Finance, Insurance, Health Care etc.}

\begin{frame}
\frametitle{Kepler's Law of Planetary Motion}
\begin{table}[]
\begin{tabular}{lll}
Planet &Distance to Sun (AU) &Period(days)\\
\hline
Mercury	&0.389	&87.77\\
Venus	&0.724	&224.70\\
Earth	&1	&365.25\\
Mars	&1.524	&686.95\\
Jupiter	&5.2	&4332.62\\
Saturn	&9.510	&10759.2\\
\hline
\end{tabular}
\end{table}
What is the mathematical model for $Period=\mathnormal{f}(DistanceSun)$?\\
Answer: $$T^2 \propto r^3$$
\end{frame}

\begin{frame}
\frametitle{Realtor Housing Price}
\begin{table}[]
\begin{tabular} { c c c c c c c}
house price &logitude & lattitude & age  &oceanProx &size &...\\ 
\hline
452600.0 &-122.23	&37.88	&41 	&NEAR BAY &85768 &...\\
358500.0 &-122.22	&37.86	&21		&NEAR BAY &40803 & ...\\
352100.0 &-122.24	&37.85	&52 	&NEAR BAY &63085 & ...\\
... & ... & ... & & & &
\end{tabular}
\end{table}

What is the mathematical model for $\text{House Pric}e=\mathnormal{f}(location, size,  ...)$?

\end{frame}

\begin{frame}
\frametitle{Predict Heart Disease}
\begin{table}[]
\begin{tabular} {c c c c c c c}
heart disease &age 	& chest pain type 	& fbs &thalach &gender &...\\ 
Yes		&63 	& 0 				&1 				&150		& F		&...\\
No 		&45 	& 1 				&0 				&170		&F		& ...\\
No 		&70 	& 0  				& 0  			&168		&M		& ...\\
No 		&30 	& 3 				&0 				&190		&F		& ...\\
Yes 	&55 	& 2 				& 0 			&148		&M		&...\\
No 		&26 	& 1 				& 1				&155	&M		&...\\
... & ... & ... & & & &
\end{tabular}
\end{table}
What is the model for $\text{heart disease}=\mathnormal{f}(age, gender, fbs,  ...)$?
\end{frame}

\section{Mathematics of Data Analytics}

\begin{frame}
\frametitle{Modeling of Structured Data}
Given a set of observations
\[ \begin{bmatrix}  \mathnormal{y}^1 \\ \mathnormal{y}^2\\\mathnormal{y}^3\\\mathnormal{y}^i \\ \vdots \\\mathnormal{y}^n \end{bmatrix}
\&
\begin{bmatrix}
   \mathnormal{x^1_1} &\mathnormal{x^1_2}  &\mathnormal{x^1_3} & \cdots &\mathnormal{x^1_m}\\
   \mathnormal{x^2_1} &\mathnormal{x^2_2} &\mathnormal{x^2_3} & \cdots &\mathnormal{x^2_m}\\
   \mathnormal{x^3_1} &\mathnormal{x^3_2} &\mathnormal{x^3_3} & \cdots &\mathnormal{x^3_m}\\
   \mathnormal{x^i_1} &\mathnormal{x^i_2} &\mathnormal{x^i_3} & \cdots &\mathnormal{x^i_m}\\
   \vdots &\vdots &\vdots & \cdots &\vdots\\
   \mathnormal{x^n_1} &\mathnormal{x^n_2} &\mathnormal{x^n_3} &\cdots &\mathnormal{x^n_m}
  \end{bmatrix}
\]
What is the best model for $\mathnormal{y}=\mathnormal{f(x_1, x_2, x_3,  ... x_m)}$, given $\mathnormal{n}$ data points of $\mathnormal{m}$-dimension?
\end{frame}


\begin{frame}
\frametitle{Linear Algerbra Review}
\begin{itemize}
\item Vectors: $\mathnormal{x=(x_1, x_2, x_3,...x_i, ..., x_m)}$
\item Dot Product: $\mathnormal{\vec{x}\cdot\vec{y}=x_1 y_1+x_2 y_2+x_3 y_3}$
\item Matrix Addition, Multiplication, 
\item Inverse $\mathnormal{{X}X^{-1}=X^{-1}X=I}$, 
\item Transpose $\mathnormal{M^{T}}$
\item Linear Combination $\mathnormal{a\vec{x}+b\vec{y}+c\vec{z}}$
\item Geometric Ineterpretation of Linear Algerbra: Linear Independent, Linearly Dependent
\end{itemize}
\end{frame}

\section{Overview of Statistic Models and Machine Learning}

\begin{frame}
\frametitle{What is $\mathnormal{f}$: Simple Linear models}

Linear Regression
$$\mathnormal{y=\beta_0+\beta_1 x_1+ \beta_2 x_2 + ... \beta_m x_m}$$

Variation
$$\mathnormal{ln(y)=\beta_0+\beta_1 x_1+ \beta_2 x_2 + ... \beta_m x_m}$$

\end{frame}


\begin{frame}
\frametitle{What is $\mathnormal{f}$: Generalized Linear Models}
When $\mathnormal{y}$ is observables of a random process and the same $\mathnormal{x_1, x_2 ... x_m}$ will leads to different $\mathnormal{y}$ i.e.

$$\mathnormal{y\sim P(x_1, x_2 ... x_m)}$$
\vspace{0.2cm}
For example: 
Logistic Regression;
Poisson Regression;
Generalized Linear Model
\end{frame}

\begin{frame}
\frametitle{What is $\mathnormal{f}$: Tree Models, Neural Network}
When close math formula does not provide good enough approximation for the problem? \\

$$\mathnormal{f}\sim
Neural network; 
Tree; 
Random Forest
$$ 
\end{frame}


\begin{frame}
\frametitle{How to find $\mathnormal{f}$: Optimization}
Training Algorithms:\\
\begin{itemize}
\item Maximum Likelihood Estimation; Entropy Maximization
\item Gradient Descent; Stochastic Gradient Descent (SGD);
\item Greedy Seach
\end{itemize}
\end{frame}


\section{Computational Tools, Library, Packages, Softwares}
\begin{frame}
\frametitle{Python Environment Setup: Demo}
\begin{itemize}
\item Python; 'pip' installtion tools
\item Packages: numpy; sklearn etc.
\item IDE: jupyter-notebook, pyCharm etc.
\item Virtual Environment
\item Reference: \url{https://github.com/DihuiLai/washu_data_analytics_foundation/blob/master/environment_setup.md}

\end{itemize}
\end{frame}

\end{document}

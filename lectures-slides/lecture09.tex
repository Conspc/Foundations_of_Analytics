%!TeX encoding = UTF-8
%!TeX program = xelatex
\documentclass[notheorems, aspectratio=54]{beamer}
% aspectratio: 1610, 149, 54, 43(default), 32
\usepackage{hyperref}

 
\usepackage{latexsym}
\usepackage{amsmath,amssymb}
\usepackage{mathtools}
\usepackage{color,xcolor}
\usepackage{graphicx}
\usepackage{algorithm}
\usepackage{amsthm}
\DeclareMathOperator*{\argmax}{arg\,max}
\DeclareMathOperator*{\argmin}{arg\,min}
\usepackage{lmodern} % 解决 font warning
% \usepackage[UTF8]{ctex}
\usepackage{animate} % insert gif

\usepackage{lipsum} % To generate test text 
\usepackage{ulem} % 下划线,波浪线

\usepackage{listings} % display code on slides; don't forget [fragile] option after \begin{frame}

% ----------------------------------------------
% tikx
\usepackage{framed}
\usepackage{fancybox}
\usepackage{tikz}
\usepackage{pgf}
\usetikzlibrary{automata, calc,trees,positioning,arrows,chains,shapes.geometric,%
    decorations.pathreplacing,decorations.pathmorphing,shapes,%
    matrix,shapes.symbols}
\pgfmathsetseed{1} % To have predictable results
% Define a background layer, in which the parchment shape is drawn
\pgfdeclarelayer{background}
\pgfsetlayers{background,main}

% define styles for the normal border and the torn border
\tikzset{
  normal border/.style={orange!30!black!10, decorate, 
     decoration={random steps, segment length=2.5cm, amplitude=.7mm}},
  torn border/.style={orange!30!black!5, decorate, 
     decoration={random steps, segment length=.5cm, amplitude=1.7mm}}}

% Macro to draw the shape behind the text, when it fits completly in the
% page
\def\parchmentframe#1{
\tikz{
  \node[inner sep=2em] (A) {#1};  % Draw the text of the node
  \begin{pgfonlayer}{background}  % Draw the shape behind
  \fill[normal border] 
        (A.south east) -- (A.south west) -- 
        (A.north west) -- (A.north east) -- cycle;
  \end{pgfonlayer}}}

% Macro to draw the shape, when the text will continue in next page
\def\parchmentframetop#1{
\tikz{
  \node[inner sep=2em] (A) {#1};    % Draw the text of the node
  \begin{pgfonlayer}{background}    
  \fill[normal border]              % Draw the ``complete shape'' behind
        (A.south east) -- (A.south west) -- 
        (A.north west) -- (A.north east) -- cycle;
  \fill[torn border]                % Add the torn lower border
        ($(A.south east)-(0,.2)$) -- ($(A.south west)-(0,.2)$) -- 
        ($(A.south west)+(0,.2)$) -- ($(A.south east)+(0,.2)$) -- cycle;
  \end{pgfonlayer}}}

% Macro to draw the shape, when the text continues from previous page
\def\parchmentframebottom#1{
\tikz{
  \node[inner sep=2em] (A) {#1};   % Draw the text of the node
  \begin{pgfonlayer}{background}   
  \fill[normal border]             % Draw the ``complete shape'' behind
        (A.south east) -- (A.south west) -- 
        (A.north west) -- (A.north east) -- cycle;
  \fill[torn border]               % Add the torn upper border
        ($(A.north east)-(0,.2)$) -- ($(A.north west)-(0,.2)$) -- 
        ($(A.north west)+(0,.2)$) -- ($(A.north east)+(0,.2)$) -- cycle;
  \end{pgfonlayer}}}

% Macro to draw the shape, when both the text continues from previous page
% and it will continue in next page
\def\parchmentframemiddle#1{
\tikz{
  \node[inner sep=2em] (A) {#1};   % Draw the text of the node
  \begin{pgfonlayer}{background}   
  \fill[normal border]             % Draw the ``complete shape'' behind
        (A.south east) -- (A.south west) -- 
        (A.north west) -- (A.north east) -- cycle;
  \fill[torn border]               % Add the torn lower border
        ($(A.south east)-(0,.2)$) -- ($(A.south west)-(0,.2)$) -- 
        ($(A.south west)+(0,.2)$) -- ($(A.south east)+(0,.2)$) -- cycle;
  \fill[torn border]               % Add the torn upper border
        ($(A.north east)-(0,.2)$) -- ($(A.north west)-(0,.2)$) -- 
        ($(A.north west)+(0,.2)$) -- ($(A.north east)+(0,.2)$) -- cycle;
  \end{pgfonlayer}}}

% Define the environment which puts the frame
% In this case, the environment also accepts an argument with an optional
% title (which defaults to ``Example'', which is typeset in a box overlaid
% on the top border
\newenvironment{parchment}[1][Example]{%
  \def\FrameCommand{\parchmentframe}%
  \def\FirstFrameCommand{\parchmentframetop}%
  \def\LastFrameCommand{\parchmentframebottom}%
  \def\MidFrameCommand{\parchmentframemiddle}%
  \vskip\baselineskip
  \MakeFramed {\FrameRestore}
  \noindent\tikz\node[inner sep=1ex, draw=black!20,fill=white, 
          anchor=west, overlay] at (0em, 2em) {\sffamily#1};\par}%
{\endMakeFramed}

% ----------------------------------------------

\mode<presentation>{
    \usetheme{CambridgeUS}
    % Boadilla CambridgeUS
    % default Antibes Berlin Copenhagen
    % Madrid Montpelier Ilmenau Malmoe
    % Berkeley Singapore Warsaw
    \usecolortheme{beaver}
    % beetle, beaver, orchid, whale, dolphin
    \useoutertheme{infolines}
    % infolines miniframes shadow sidebar smoothbars smoothtree split tree
    \useinnertheme{circles}
    % circles, rectanges, rounded, inmargin
}
% 设置 block 颜色
\setbeamercolor{block title}{bg=red!30,fg=white}

\newcommand{\reditem}[1]{\setbeamercolor{item}{fg=red}\item #1}

% 缩放公式大小
\newcommand*{\Scale}[2][4]{\scalebox{#1}{\ensuremath{#2}}}

% 解决 font warning
\renewcommand\textbullet{\ensuremath{\bullet}}

% ---------------------------------------------------------------------
% flow chart
\tikzset{
    >=stealth',
    punktchain/.style={
        rectangle, 
        rounded corners, 
        % fill=black!10,
        draw=white, very thick,
        text width=6em,
        minimum height=2em, 
        text centered, 
        on chain
    },
    largepunktchain/.style={
        rectangle,
        rounded corners,
        draw=white, very thick,
        text width=10em,
        minimum height=2em,
        on chain
    },
    line/.style={draw, thick, <-},
    element/.style={
        tape,
        top color=white,
        bottom color=blue!50!black!60!,
        minimum width=6em,
        draw=blue!40!black!90, very thick,
        text width=6em, 
        minimum height=2em, 
        text centered, 
        on chain
    },
    every join/.style={->, thick,shorten >=1pt},
    decoration={brace},
    tuborg/.style={decorate},
    tubnode/.style={midway, right=2pt},
    font={\fontsize{10pt}{12}\selectfont},
}
% ---------------------------------------------------------------------

% code setting
\lstset{
    language=C++,
    basicstyle=\ttfamily\footnotesize,
    keywordstyle=\color{red},
    breaklines=true,
    xleftmargin=2em,
    numbers=left,
    numberstyle=\color[RGB]{222,155,81},
    frame=leftline,
    tabsize=4,
    breakatwhitespace=false,
    showspaces=false,               
    showstringspaces=false,
    showtabs=false,
    morekeywords={Str, Num, List},
}

% ---------------------------------------------------------------------

%% preamble
\title{Lecture 09}
% \subtitle{The subtitle}
\author{Dihui Lai}
\institute[WUSTL]{dlai@wustl.edu}

% -------------------------------------------------------------

\begin{document}

%% title frame
\begin{frame}
    \titlepage
\end{frame}
\begin{frame}
\frametitle{Part of Speech (POS) Tagging}
\begin{itemize}
\item General Part-of-speech (POS) categories: 
\begin{itemize}
\item nouns (data, chair),
\item verbs (debate, discuss),
\item  adverbs (slowly), 
\item  adjectives (funny)
\item etc.
\end{itemize}
\item Fine-grained POS categories:
\begin{itemize}
\item nouns: proper noun (Lisa, Paris) v.s. common nouns (woman, city); 
\item adverb: comparative (more happily) v.s. superlative (most happily)
\item etc.
\end{itemize}
\item A word could have multiple POS tags e.g. 
\begin{itemize}
\item A number of factors account [verb] for the differences between the two scores.
\item How do I open an account [noun] with your bank?
\end{itemize}
\end{itemize}
\end{frame}

\begin{frame}

\frametitle{English Penn Treebank part-of-speech Tagset}

An important tagset for English is the 45-tag Penn Treebank tagset. 

\vspace{1cm}

\url{https://www.ling.upenn.edu/courses/Fall_2003/ling001/penn_treebank_pos.html}

\end{frame}

\begin{frame}
\frametitle{POS-tag Model}

\begin{itemize}
\item Label the words in a document using POS tags, e.g. 

The[DT] Itek[NNP] Air[NNP] Boeing[NNP] 737[CD] took[VBD] off[RP] bound[VBN] for[IN] Mashhad[NNP] in[IN] north-eastern[JJ] Iran[NNP].

\item If a word $w$ that could be tagged as $\mathnormal{t_1}$, $\mathnormal{t_2}$, ... $\mathnormal{t_k}$, the probabilities the word has tagged  $\mathnormal{t_i}$ is calculated as
$\mathnormal{p(t_i|w) = \frac{c(w,t_i)}{\sum\limits_{i=1}^{k}c(w, t_i)}}$

\textbf{This approach does not take the order of the word into consideration!}
\end{itemize}

\end{frame}

\begin{frame}
\frametitle {POS tag Sequencing Model}
Provided that we have a sequence of words $\mathnormal{W=w_1, w_2,  ...w_i, ...w_n}$ and we want to figure out the their POS tags $\mathnormal{T= t_1, t_2, ... t_i ...t_n}$

Using Bayes' theorem
$$\mathnormal{P(T | W) = P(W|T) P(T) / P(W) = const \times P(W|T) P(T)}$$

Assume that $\mathnormal{t_i}$ is only dependent on $\mathnormal{t_{i-1}}$ and $\mathnormal{w_{i}}$, we have
\begin{align*}
\mathnormal{P(T)}&=\mathnormal{ P(t_1) P(t_2 | t_1) P(t_3 | t_1, t_2) P(t_4 | t_1, t_2, t_3)  ...P(t_n | t_{1}, t_{2}, ... t_{n-1})}\\
&=\mathnormal{ P(t_1) P(t_2 | t_1) P(t_3 | t_2) P(t_4 |  t_3)  ...P(t_n | t_{n-1})}
\end{align*}
\begin{align*}
\mathnormal{P(W|T)}=\mathnormal{ P(w_1 | t_1) P(w_2 | t_2) P(w_3 | t_3)  ...P(w_n | t_{n})}
\end{align*}
\end{frame}

\begin{frame}
\frametitle {POS tag Sequencing Model (Continued)}
In summary, we have 
$$\mathnormal{P(T|W) \approx P(t_1) P(t_2|t_1) ... P(t_{n}|t_{n-1}) P(w_1|t_1) P(w_2|t_2) ... P(w_n|t_n)}$$
Each term on the right hand side of the equation can be calculated as 
$$\mathnormal{P(t_{i}|t_{i-1}) = \frac{c(t_{i-1},t_{i})}{c(t_{i-1})}} \text{ (transition probability)}$$ , 
$$\mathnormal{P(w_i|t_i) = \frac{c(w_i,t_i)}{c(t_i)}} \text{ (emission probability)}$$

where 

$\mathnormal{c(t_i)} = \text{count of } \mathnormal{t_i} \text{ in the corpus}$,  

$ \mathnormal{c(w_i,t_i)} = \text{count of }  \mathnormal{(w_i, t_i)} \text{ in the corpus}$,

$ \mathnormal{c(t_{i-1},t_{i})} = \text{count of }  \mathnormal{(t_{i-1}, t_i)} \text{ in the corpus}$

\end{frame}

\begin{frame}
\frametitle {POS tag Sequencing Model (Continued): Hidden Markov Model and Viterbi Algorithm}
The best tag sequence is the sequence that maximize the conditional probability $P(T|W)$ i.e.
\begin{align*}
\mathnormal{(t_1, t_2, ...t_n)} & = \mathnormal{\argmax_{t_1, t_2, ...t_n}P(t_1, t_2..., t_n|w_1, w_2, ...w_n)}\\
&= \mathnormal{\argmax_{t_1, t_2, ...t_n} \prod\limits_{i=1}^n P(w_i |t_i ) P(t_i |t_{i-1})}
\end{align*}

To solve the problem, we can use the Viterbi Algorithms for Hidden Markov Model
\end{frame}

\begin{frame}
\frametitle {POS tag Model: the State of the Art}

\url{https://aclweb.org/aclwiki/POS_Tagging_(State_of_the_art)}
\end{frame}

\begin{frame}
\frametitle{Name Entity Recognition (NER)}
\begin{itemize}
\item POS tag usually label a single word. However, to understand text, we also need to consider the meaning of a set of words.
\item NER labels words in a texts that are names of things e.g. person, organization, money amount, gene/protein names
\item John (person) Lee (person) is the chief of CBSE (organization). 
\item the output of POS-tag could be used as input to accomplish a NER model.
\end{itemize}
\end{frame}

\begin{frame}
\frametitle{NER and IOB Format Tag}
IOB: I-Inside, O-Outside, B-Begin

Example: Alex is going to Los Angeles
\begin{align*}
&\text{Alex I-PER}\\
&\text{is O}\\
&\text{going O}\\
&\text{to O}\\
&\text{Los B-LOC}\\
&\text{Angeles I-LOC}
\end{align*}
\end{frame}

\begin{frame}
\frametitle{NER model and the State of the Art}
\begin{itemize}
\item The state of art algorithms:
\url{https://paperswithcode.com/sota/named-entity-recognition-ner-on-conll-2003}
\item Useful packages, softwares, services
\url{https://nlp.stanford.edu/software/CRF-NER.shtml}
\url{http://nlp_architect.nervanasys.com/installation.html}
\url{https://cloud.google.com/natural-language/}

\end{itemize}


\end{frame}

\end{document}
